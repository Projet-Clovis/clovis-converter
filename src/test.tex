\documentclass{article}%
\usepackage[T1]{fontenc}%
\usepackage[utf8]{inputenc}%
\usepackage{lmodern}%
\usepackage{textcomp}%
\usepackage{lastpage}%
\usepackage{tcolorbox}%
\usepackage{xcolor}%
%
\title{Algorithmes d'Optimisation des Graphes}%
\author{Licence 3}%
\date{2021 {-} 2022}%
\normalsize%

\definecolor{danger}{HTML}{e6505f}
\definecolor{danger-bg}{HTML}{fce5e7}
%
\begin{document}%
\normalsize%
\maketitle%
\tableofcontents%
\newpage%
\section{Le sujet}%
\label{sec:Lesujet}%

%
\subsection{Problématiques}%
\label{subsec:Problmatiques}%
Existe{-}t{-}il réellement un "moi{-}même", cette identité que je revendique ?%
\newline%
\newline%
Si elle existe, cette identité est{-}elle immuable ?%
\newline%
\newline%
Et est{-}elle nichée en moi ou n'est{-}elle qu'une apparence extérieure ?%
\newline%
\newline%
Définitions à connaître : éducation, alter ego, schizophrénie.

%
\subsection{Le sujet selon Descartes}%
\label{subsec:LesujetselonDescartes}%
Descartes donne un définition universelle du sujet, valable pour tous les humains. Le seul élément indiscutable auquel il parvient est qu'il est sûr de penser. Une autre certitude émerge : s'il pense, c'est qu'il existe. Donc si la pensée existe, l'entité qui l'exprime existe aussi. Même si elle s'illusionne sur ce qui fait le réel. Pour Descartes, le sujet est donc à l'origine des pensées. ce sujet derrière la pensée est appelé sujet pensant ou être pensant. La métaphysique de Descartes repose sur le cogito ergo sum, c'est{-}à{-}dire, "Je pense donc je suis".

\begin{tcolorbox}[colframe=danger, colback=danger-bg]
{ \footnotesize \textcolor{danger}{ATTENTION}}
\\ Par le désir, l'Homme fait l'expérience du manque et de la souffrance. Souvent cependant, les hommes sont faibles lorsqu'il s'agit de réprimer ou de contrôler leurs désirs.
\end{tcolorbox}

\begin{tcolorbox}[width=5cm]
Here is some text
\end{tcolorbox}

\begin{tcolorbox}[width=.5\textwidth, colframe=red]
Here is some text
\end{tcolorbox}

\begin{tcolorbox}[width=8cm, colframe=red, colback=blue!30, halign=right]
Here is some text
\end{tcolorbox}

\begin{tcolorbox}[width=.5\linewidth, halign=center, colframe=red, colback=blue!30, boxsep=5mm, arc=3mm]
Here is some text
\end{tcolorbox}

\begin{tcolorbox}[width=7cm, colframe=red, colback=blue!30, arc=3mm, sharp corners=east]
Here is some text
\end{tcolorbox}

%
\end{document}
