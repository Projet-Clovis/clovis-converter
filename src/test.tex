\documentclass{article}%

% -------------------- Packages --------------------
\usepackage[T1]{fontenc}%
\usepackage[utf8]{inputenc}%
\usepackage{lmodern}%
\usepackage{textcomp}%
\usepackage{lastpage}%
\usepackage{xcolor}%

\usepackage[framemethod=tikz]{mdframed}
\usepackage{tikzpagenodes}
\usetikzlibrary{calc}

\usepackage{fontawesome5}
%\usepackage{soul}% letter spacing


% -------------------- Title --------------------
\title{Algorithmes d'Optimisation des Graphes}%
\author{Licence 3}%
\date{2021 {-} 2022}%
\normalsize%


% -------------------- Colors --------------------
\definecolor{danger}{HTML}{e6505f}
\definecolor{danger-bg}{HTML}{fce5e7}
\definecolor{summary}{HTML}{e8ae95}
\definecolor{summary-bg}{HTML}{fcf3ef}
\definecolor{definition}{HTML}{2f80ed}
\definecolor{definition-bg}{HTML}{e0ecfd}

\definecolor{code-bg}{HTML}{2d334b}

\definecolor{border-color}{HTML}{dadce0}



% -------------------- Styles --------------------
\mdfdefinestyle{mystyle}{%
  innertopmargin=10px,
  innerbottommargin=10px,
  linecolor=danger,
  backgroundcolor=danger-bg,
  roundcorner=4px
}
\newmdenv[style=mystyle]{solution}


% -------------------- Document --------------------
\begin{document}%
\normalsize%
\sffamily
\maketitle%
\tableofcontents%
\newpage%
\section{Le sujet}%
\label{sec:Lesujet}%

%
\subsection{Problématiques}%
\label{subsec:Problmatiques}%
Existe{-}t{-}il réellement un "moi{-}même", cette identité que je revendique ?%
\newline%
\newline%
Si elle existe, cette identité est{-}elle immuable ?%
\newline%
\newline%
Et est{-}elle nichée en moi ou n'est{-}elle qu'une apparence extérieure ?%
\newline%
\newline%
Définitions à connaître : éducation, alter ego, schizophrénie.

%
\subsection{Le sujet selon Descartes}%
\label{subsec:LesujetselonDescartes}%
Descartes donne un définition universelle du sujet, valable pour tous les humains. Le seul élément indiscutable auquel il parvient est qu'il est sûr de penser. Une autre certitude émerge : s'il pense, c'est qu'il existe. Donc si la pensée existe, l'entité qui l'exprime existe aussi. Même si elle s'illusionne sur ce qui fait le réel. Pour Descartes, le sujet est donc à l'origine des pensées. ce sujet derrière la pensée est appelé sujet pensant ou être pensant. La métaphysique de Descartes repose sur le cogito ergo sum, c'est{-}à{-}dire, "Je pense donc je suis".

\begin{solution}
{ \scriptsize \textcolor{danger}{\faIcon{exclamation-triangle} \textbf{ATTENTION}}}
\vspace{3px}
\\ Par le désir, l'Homme fait l'expérience du manque et de la souffrance. Souvent cependant, les hommes sont faibles lorsqu'il s'agit de réprimer ou de contrôler leurs désirs.
\end{solution}

\begin{solution}[linecolor=summary, backgroundcolor=summary-bg]
{ \scriptsize \textcolor{summary}{\faIcon{graduation-cap} \textbf{RÉSUMÉ}}}
\vspace{3px}
\\ Par le désir, l'Homme fait l'expérience du manque et de la souffrance. Souvent cependant, les hommes sont faibles lorsqu'il s'agit de réprimer ou de contrôler leurs désirs.
\end{solution}

\begin{solution}[linecolor=definition, backgroundcolor=definition-bg]
{ \scriptsize \textcolor{definition}{\faIcon{graduation-cap} \textbf{RÉSUMÉ}}}
\vspace{3px}
\\ \underline{\textbf{Java Runtime Environment}}
\vspace{2.5px}
\\ \textit{Environnement d'exécution Java}, qui crée la machine virtuelle JVM.
\end{solution}

\begin{mdframed}[topline=false, bottomline=false, linecolor=border-color]
\begin{center}
    Malheur à qui n'a plus rien à désirer !
    \\ Rousseau
\end{center}
\end{mdframed}

\begin{solution}[linecolor=code-bg, backgroundcolor=code-bg]
\textcolor{white}{\texttt{npm install puppeteer}}
\end{solution}

%
\end{document}
